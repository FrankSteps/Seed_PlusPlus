\documentclass[12pt,a4paper]{article}
\usepackage[utf8]{inputenc}
\usepackage[brazil]{babel}
\usepackage{graphicx}
\usepackage{hyperref}
\usepackage{geometry}
\geometry{margin=2cm}
\usepackage{lipsum} % Para gerar Lorem Ipsum

\title{Documentação do Projeto Seed++}
\author{Francisco Passos}
\date{\today}
\begin{document}

\maketitle
\tableofcontents
\newpage

\section{Introdução à documentação}
\lipsum[1-2]

%====================================================
\section{Manual do usuário}

\subsection{Modo leitura}
\lipsum[3]

\subsection{Modo Administrador}
\lipsum[4]

\subsubsection{Cadastrar uma nova digital}
\lipsum[5]

\subsubsection{Apagar uma digital em específica}
\lipsum[6]

\subsubsection{Apagar todas as digitais do Seed++}
\lipsum[7]

\subsection{Para situações de emergência}
\lipsum[8]

%====================================================
\section{Manual do Desenvolvedor}

\subsection{Introdução base}
\lipsum[9]

%----------------------------------------------------
\subsection{Hardware}

\subsubsection{Peças usadas}
\lipsum[10]

\subsubsection{Diagrama esquemático}
\lipsum[11]

\subsubsection{Orientações básicas}
\lipsum[12]

%----------------------------------------------------
\subsection{Software}

\subsubsection{Introdução}
\lipsum[13]

\subsubsection{Arquitetura do Software}

\paragraph{Bibliotecas utilizadas}
\lipsum[14]

\paragraph{Funções principais}
\lipsum[15]

\paragraph{Módulos do sistema}
\lipsum[16]

\subsubsection{Fluxo de Operação}
\lipsum[17]

\subsubsection{Interação com o Hardware}
\lipsum[18]

\subsubsection{Modos de Operação}

\paragraph{Modo Administrador (ADM)}
\lipsum[19]

\paragraph{Modo Leitura}
\lipsum[20]

\subsubsection{Tratamento de Erros e Feedback do Usuário}
\lipsum[21]

\subsubsection{Boas Práticas e Segurança}
\lipsum[22]

\end{document}
